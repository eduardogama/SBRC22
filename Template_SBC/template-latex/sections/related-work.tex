\section{Related Work}
% A multi-tier fog content orchestrator mechanism with quality of experience support


This section presents the related papers area.
%Esta seção apresenta os trabalhos relacionados da area.
%ES-HAS: An Edge- and SDN-Assisted Framework for HTTP Adaptive Video Streaming
In Farahani \textit{et al.}~[1], the paper propose here an optimization framework to implement an optimal serving strategy. Second, it assists the clients when they do their requests so that the number of requests that should be forwarded to the origin server is minimized.

%Can Accurate Future Bandwidth Prediction Improve Volumetric Video Streaming Experience?
In Bentaleb \textit{et al.}~[2], The solution implements a deep-neural-network solution to predict  the multimedia content to be consumed, this way, the edge server may request antecipated to the cloud volumetric video segments.

% Multimedia Microservice Placement in Hierarchical Multi-tier Cloud-to-Fog Networks
In Santos \textit{et al.} [3], his work presented a process to design/create a hierarchical multi-tier Cloud-to-Fog network for multimedia services distribution.

% QoE Ready to Respond: A QoE-aware MEC Selection Scheme for DASH-based Adaptive Video Streaming to Mobile Users
In Shi \textit{et al.} [4], this paper introduces QoE Ready to Respond (QoE-R2R), a QoE-aware MEC Selection scheme for DASH-based mobile adaptive video streaming for optimizing video transmis- sion in a MEC-supported network environment.

% H2BR: An HTTP/2-based Retransmission Technique to Improve the QoE of Adaptive Video Streaming
In Nguyen \textit{et al.} [5], a retransmission mechanism based on  HTTP/2 is deployed to improve the QoE, the throughput perceveid by the users is coomputed while the segments are received, the server retransmit segments which are already in the client buffer, but in higher-quality versions.

%A multi-tier fog content orchestrator mechanism with quality of experience support
In Santos \textit{et al.} [6] present an orchestration mechanism aimed at the selection of Fog nodes for video content download. The mechanism receives user feedback to assess the video streaming from the Fog nodes.

To handle resource management is an essential requirement in a well-constructed archestration. Within the highlighted articles, only Shi deals with aspects directly related to wireless network issues, where the content displacement to the edge of the network does not necessarily and the connection between cache and and bs operate separately. However, dealing with a multi-level architecture brings aspects that can help or worsen the network performance if not handled correctly. Thus, These challenges call for the make-up and development of new orchestration are addressed, an architecture for the multi-tier edge network to handle such strict problems are presented.