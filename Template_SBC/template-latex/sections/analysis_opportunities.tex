\section{Analysis and Opportunities}
\label{sec:analysis_opportunities}

Para projetar uma arquitetura de entrega de video em multiníveis,

This section describes the opportunities in multi-tier edge networks and motivates the opportunities with the result analysis of the FiCloud article.

Designing a cache hierarchy on vertically organized edge
nodes with an arbitrary number of tiers can present improve-
ments in users’ QoE [13]. The architecture mentioned above
works toward such advantages by serving the requested con-
tent as close as possible to the end-user, efficiently forwarding
requests between parent edge nodes within the hierarchy, and
balancing the video traffic considering hop counts and users
attended. In addition, the network core congestion is reduced
since it represents an operational overhead for the contentprovider. As a preliminary outcome achieved by a multi-tier
network experiment, the QoE impact over a video streaming
service is assessed. After that, we describe some results about
the QoE characteristics and insights on the opportunities
of caching multimedia content in edge nodes of multi-tier
networks.


\subsection{Impact of Fog Multi-tier Network Approach}

To illustrate the differences in users’ performance requesting
a video from cache nodes in different tiers, consider two
users requesting the same multimedia content from different
layers.

Based on these observations, a simple strategy of moving
the video to the edge can significantly improve the user’s QoE.
In this way, the video transmission system can provide user
satisfaction qualities and keep them watching the video up
to the end. However, if there is no correct management of
connections in real-time and a dynamic mechanism to tackle
with a varying load coming, for example, from the mobility
of users, we can conclude that the impact introduced by the
AP changes can significantly decrease their QoE. The user
experience can end up getting worse even using the edge of
the network. Proper management of multimedia content can be
done in which the VoD focuses on providing a better QoE for
the users’ connection changes. A content migration mechanism
at the edge and upper tiers can mitigate the problem. Also,
performing a rerouting between the active users’ connections
and the server nodes may help in improving and balance user’s
QoE.