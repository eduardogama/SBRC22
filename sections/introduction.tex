\section{Introduction}
\label{sec:intro}

%******* Introduction of the Dash technology and Cloud/Fog networks****
Atualmente, os conteúdos multimídia são fornecidos a uma infinidade de dispositivos, onde a exigência pela qualidade de imagem muda de acordo com a dispositivos, que vão de pequenos telefones celulares a TVs de alta resolução. Para lidar com uma transmissão de video com diferentes e rigorosos requisitos, como um canal de comunicação de boa qualidade bem como um fluxo constante e ininterrupto de informações~\cite{Immich2018WinNet}. Essa demanda deve manter uma boa Qualidade de Experiencia~(\textit{Quality of Experience} - QoE) do usuário. Hoje, o tráfego multimídia corresponde à maior parte dos dados que circulam pela Internet em todo o mundo, e tal previsão de crescimento se mantém  para os próximos anos. 
Dentro desta rede o protocolo utilizado pelos grandes players do mercado (como Microsoft, Apple, Adobe e Netflix) adotam o paradigma Transmissão Adaptativa HTTP~(\textit{HTTP Adaptive Streaming} - HAS)~\cite{company:dashs}. Hoje, eles são decompostos essencialmente em Servidores HTTP para atender essa demanda.

o cliente e o servidor utilizam o protocolo HTTP na camada de aplicação para realizar todas as requisições/respostas necessárias. No lado do servidor, uma vez que um arquivo de mídia~(ou fluxo) esteja pronto, ele será preparado para streaming antes de ser publicado em um servidor HTTP padrão. O arquivo/stream original é particionado em segmentos~(também chamados \textit{chunks}) de tempo de reprodução equivalente, e várias versões (também chamadas de representações) de cada segmento são geradas que variam em taxa de bits/resolução/qualidade usando um codificador ou um transcodificador.
A lista as representações disponíveis incluem informações como tempo de vídeo, disponibilidade de conteúdo, tipos de mídia~(ou seja, H.264 , H.265, etc. .), resoluções, larguras de banda mínimas e máximas, e a existência de várias alternativas codificadas de componentes multimídia, localização de segmentos de mídia na rede e outras características de conteúdo.

No lado do cliente HAS, o player solicita os segmentos de video sequencialmente e adapta-se dinamicamente às condições da rede usando sua lógica adaptativa de taxa de bits~(ABR). Os esquemas ABR também levam em consideração o buffer de reprodução, recursos do dispositivo, preferências do visualizador, bem como recursos de conteúdo, com pesos diferentes.
Como a QoE do espectador precisa ser determinada em tempo real durante a reprodução, métricas objetivas são frequentemente usadas, incluindo o número de interrupções, duração do atraso de inicialização, frequência e número de oscilações de qualidade de vídeo. Por padrão, o HAS não requer nenhum esquema de adaptação específico, deixando os desenvolvedores de sistemas inovarem e implementarem seus próprios métodos.

% ----------------------------------------------------------------------------
% What is the current context in edge-cloud network in terms of traffic and users? 
% ----------------------------------------------------------------------------
Hoje, as soluções existentes para serviços de streaming de vídeo são projetadas para fornecer conteúdo multimídia em data centers de edge-cloud. Esses sistemas de adaptação baseados em DASH normalmente são capazes de se adequar às condições de entrega de rede tanto na borda quanto na nuvem, o que traz recursos de computação para a borda da rede e, portanto, mais perto dos usuários~\cite{sitaraman:ACD2014}. Dessa forma, a criação desses sistemas resolve parcialmente os problemas de escalabilidade, disponibilidade e interoperabilidade, mas ao mesmo tempo apresenta novos desafios~(por exemplo, maior latência e congestionamento da rede principal)~\cite{tran:wons17,ye: ITC17, taleb:JSAC18}. Vários trabalhos na literatura destacam o fog/edge computing para lidar com as novas demandas de tráfego de vídeo que estão surgindo. Onde data centers com menos capacidade de processamento e armazenamento podem fornecer serviços/virtualização mais próximos do usuário final. Assim, a borda da rede pode fornecer taxas de latência que a nuvem não conseguiria de outra forma~\cite{gamaUCC2019, rosarioSENSORS2018}.

% ----------------------------------------------------------------------------
% What carries have been doing to address the increasing traffic?
% ----------------------------------------------------------------------------

Infelizmente, o uso do MEC aproxima o conteúdo dos usuários móveis, mas também introduz o problema da seleção do MEC. O MEC de serviço pode não ser aquele implantado na BS de serviço. O handover fará com que o usuário se comunique com uma BS alvo, mas não trará a conexão para um MEC vizinho. Portanto, os provedores de conteúdo ou operadoras de rede precisam encontrar um MEC apropriado para cada usuário móvel. Existe a necessidade de um processo de seleção do MEC que deve ser realizado em tempo hábil, a fim de apoiar a alta qualidade do serviço. Um mecanismo de migração de conteúdo na borda e nas camadas superiores pode atenuar o problema. Além disso, realizar um reencaminhamento entre as conexões dos usuários ativos e os nós do servidor pode ajudar a melhorar e equilibrar a QoE do usuário.


Embora muitos trabalhos de pesquisa abordem serviços de streaming de vídeo em conjunto com \textit{fog/cloud computing}, existem aspectos pouco abordados nas soluções atuais~\cite{Mouradian2018ComSurv, bentaeb:2018:MSys}. Geralmente, as arquiteturas de streaming de vídeo buscam diminuir a carga de tráfego e melhorar a QoE da entrega de vídeo. Tais soluções não levam em consideração o comportamento do player de vídeo utilizado pelo usuário, bem como aspectos relacionados à mobilidade do usuário nos mecanismos de tomada de decisão em ambientes multiníveis. A fim de abordar as questões acima, este trabalho tem como objetivo modelar mecanismos de entrega de vídeo em DASH para serem utilizados em ambientes de Smart City. Tais mecanismos propostos aproveitarão tecnologias emergentes relacionadas a redes~ (como computação de borda/nuvem e microsserviços), com o objetivo de auxiliar a tomada de decisão do serviço de streaming de vídeo em uma infraestrutura urbana.